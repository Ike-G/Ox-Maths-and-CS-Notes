\documentclass{tikzposter} %Options for format can be included here
\geometry{paperwidth=850mm, paperheight=1500mm}
\makeatletter
\setlength{\TP@visibletextwidth}{\textwidth-2\TP@innermargin}
\setlength{\TP@visibletextheight}{\textheight-2\TP@innermargin}
\makeatother
\usepackage{amsmath}
\usepackage{parskip}
\usepackage{amsfonts}
\usepackage{amssymb}
\usepackage{verbatim}
\usepackage{nicefrac}
\usepackage{mathtools}
\DeclarePairedDelimiter{\floor}{\lfloor}{\rfloor}
\DeclarePairedDelimiter{\ceil}{\lceil}{\rceil}
\DeclareMathOperator{\Log}{Log}
\DeclareMathOperator{\res}{res}
\newcommand\restr[2]{{% we make the whole thing an ordinary symbol
    \left.\kern-\nulldelimiterspace % automatically resize the bar with \right
      #1 % the function
      \vphantom{\big|} % pretend it's a little taller at normal size
    \right|_{#2} % this is the delimiter
  }}

\newtheorem{theorem}{Theorem}
\newtheorem{lemma}[theorem]{Lemma}
\newtheorem{corollary}{Corollary}

\newtheorem{definition}{Definition}

\newtheorem{remark}{Remark}
\newtheorem{claim}{Claim}
\newtheorem{case}{Case}

\definecolor{nbYellow}{HTML}{FCF434}
\definecolor{nbPurple}{HTML}{9C59D1}
\definecolor{nbBlack}{HTML}{2C2C2C}
\definecolor{tBlue}{HTML}{5BCEFA}
\definecolor{tPink}{HTML}{F5A9B8}
\definecolor{bp1}{HTML}{D60270}
\definecolor{bp2}{HTML}{9B4F96}
\definecolor{bp3}{HTML}{0038A8}
\definecolor{pcs1}{HTML}{B300B3}
\definecolor{pcs2}{HTML}{54007D}
\definecolor{pcs3}{HTML}{B30086}
\definecolor{pcs4}{HTML}{3C00B3}
\definecolor{pcs5}{HTML}{2A007D}

\definecolorstyle{NewColour} {
  \definecolor{c1}{named}{nbBlack}
  \definecolor{c2}{named}{nbPurple}
  \definecolor{c3}{named}{nbYellow}
}{
  % Background Colors
  \colorlet{backgroundcolor}{black!10}
  \colorlet{framecolor}{black}
  % Title Colors
  \colorlet{titlefgcolor}{black}
  \colorlet{titlebgcolor}{black!10}
  % Block Colors
  \colorlet{blocktitlebgcolor}{c1}
  \colorlet{blocktitlefgcolor}{white}
  \colorlet{blockbodybgcolor}{white}
  \colorlet{blockbodyfgcolor}{black}
  % Innerblock Colors
  \colorlet{innerblocktitlebgcolor}{c2!80}
  \colorlet{innerblocktitlefgcolor}{black}
  \colorlet{innerblockbodybgcolor}{c2!50}
  \colorlet{innerblockbodyfgcolor}{black}
  % Note colors
  \colorlet{notefgcolor}{black}
  \colorlet{notebgcolor}{c3!50}
  \colorlet{notefrcolor}{c3!70}
}

\defineblockstyle{NewBlock}{
  titlewidthscale=1, bodywidthscale=1, titleleft,
  titleoffsetx=0pt, titleoffsety=0pt, bodyoffsetx=0pt, bodyoffsety=0pt,
  bodyverticalshift=0pt, roundedcorners=0, linewidth=0pt, titleinnersep=1cm,
  bodyinnersep=1cm
}{
  \ifBlockHasTitle%
    \draw[draw=none, fill=blocktitlebgcolor]
    (blocktitle.south west) rectangle (blocktitle.north east);
  \fi%
  \draw[draw=none, fill=blockbodybgcolor] %
  (blockbody.north west) [rounded corners=30] -- (blockbody.south west) --
  (blockbody.south east) [rounded corners=0]-- (blockbody.north east) -- cycle;
}

% Choose Layout
\usecolorstyle{NewColour}
\usebackgroundstyle{Default}
\usetitlestyle{Filled}
\useblockstyle{NewBlock}
\useinnerblockstyle[roundedcorners=0.2]{Default}
\usenotestyle[roundedcorners=0]{Default}

\settitle{\centering \color{titlefgcolor} {\Large \@title \, -- \, \@author}}

% Title, Author, Institute
\title{Mathematical Modelling in Biology}
\author{Ike Glassbrook}

\begin{document}

% Title block with title, author, logo, etc.
\maketitle[titletoblockverticalspace=0.4cm]
\block{Continuous-time models for a single species}{
  In order to model population dynamics, we establish a law governing the conservation of population:
    \begin{align*}
      \text{rate of increase of population } = &\text{birth rate} - \text{death rate} \\
      &\text{rate of immigration} - \text{rate of emigration}.
    \end{align*}
    We generally assume in this:
    \begin{itemize}
            \item \ The system is closed, so there is no immigration, and no emigration.
            \item \ There is no spatial dependence. This is the \emph{well-mixed assumption}.
            \item \ Time is continuous.
    \end{itemize}
    With the population density function $N : [0,\infty) \to [0,\infty)$, we can write the population conservation law as
    \begin{align*}
      \frac{\mathrm{d}N}{\mathrm{d}t} = Ng(N)
    \end{align*}
    where $g$ is the net intrinsic per growth rate for population $N$. We can take this as relatively reasonable in the abstract, because in practice we would hope that our model is invariant to time-shifts. \\

    A basic example of this is the Malthus model, which takes $g(N) = b-d$, with $b$, $d$ constant birth and death rates respectively. This gives us $N(t) = N_{0}e^{(b-d)t}$. This doesn't really make sense in the long-term, although we might view it as accurate for small populations (or even as locally accurate). \\

    The Verhulst model adapts this slightly, suggesting that $g$ ought to be a decreasing function in $N$:
    \begin{align*}
      g(N) = r\left(1-\frac{N}{K}\right).
    \end{align*}
    We label $K$ the `carrying capacity'. This gives the equation
    \begin{align*}
      \int \frac{N'}{N(1-N/K)} \,\mathrm{d}t &= \int \frac{N'}{N} \,\mathrm{d}t + \int \frac{N'}{K-N} \,\mathrm{d}t \\
                                          &= \log \frac{N}{K-N} + \log N_{0} \\
      &= rt,
    \end{align*}
    from which one can derive the result. \\

    A key point of consideration when looking at a model is the steady states, and their stability. A steady state, in general, is a point where the dynamics do not change in time. Thus when we have differentiable functions, these are just the stationary points. A \emph{stable} steady state is one where beginning at the steady state means the system remains close to the steady state across time. More rigorously, a steady state $N_{s}$ is stable if and only if, for all $t > 0$, $\varepsilon > 0$ there is a $\delta$ such that $|N_{0} - N_{s}| < \delta$ means $|N_{N_{0}}(t)-N(s)| < \varepsilon$. \\

    If $f'(N_{s}) < 0$, then $1/|f'(N_{s})|$ is called the recovery time -- the time taken for an initial pertubation from the steady state to decrease by a factor of $e$. \\

    \innerblock{Nondimensionalisation and Hysteresis}{
      First, we consider the Spruce budworm model. Let $N(t)$ be equal to the budworm density, and
      \begin{align*}
        \frac{\mathrm{d}N}{\mathrm{d}t} = r_{B}N\left(1-\frac{N}{K_{B}}\right)
      \end{align*}
      In addition, suggest that there are predators, and that we model this simply as a subtracted term $p(N)$. The simplest thing would be to have $p(N)$
    }
}

\end{document}
