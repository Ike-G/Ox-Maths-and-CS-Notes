\message{ !name(main.tex)}\documentclass{tikzposter} %Options for format can be included here
\geometry{paperwidth=850mm, paperheight=1200mm}
\makeatletter
\setlength{\TP@visibletextwidth}{\textwidth-2\TP@innermargin}
\setlength{\TP@visibletextheight}{\textheight-2\TP@innermargin}
\makeatother
\usepackage{amsmath}
\usepackage{dsfont}
\usepackage{parskip}
\usepackage{amsfonts}
\usepackage{amssymb}
\usepackage{verbatim}
\usepackage{nicefrac}
\usepackage{mathtools}
\DeclarePairedDelimiter{\floor}{\lfloor}{\rfloor}
\DeclarePairedDelimiter{\ceil}{\lceil}{\rceil}
\DeclareMathOperator{\Int}{int}
\DeclareMathOperator{\cl}{cl}
\newcommand\restr[2]{{% we make the whole thing an ordinary symbol
  \left.\kern-\nulldelimiterspace % automatically resize the bar with \right
  #1 % the function
  \vphantom{\big|} % pretend it's a little taller at normal size
  \right|_{#2} % this is the delimiter
  }}

\newtheorem{theorem}{Theorem}
\newtheorem{lemma}[theorem]{Lemma}
\newtheorem{corollary}{Corollary}

\newtheorem{definition}{Definition}

\newtheorem{remark}{Remark}
\newtheorem{claim}{Claim}
\newtheorem{case}{Case}

\definecolor{nbYellow}{HTML}{FCF434}
\definecolor{nbPurple}{HTML}{9C59D1}
\definecolor{nbBlack}{HTML}{2C2C2C}
\definecolor{tBlue}{HTML}{5BCEFA}
\definecolor{tPink}{HTML}{F5A9B8}
\definecolor{bp1}{HTML}{D60270}
\definecolor{bp2}{HTML}{9B4F96}
\definecolor{bp3}{HTML}{0038A8}
\definecolor{pcs1}{HTML}{B300B3}
\definecolor{pcs2}{HTML}{54007D}
\definecolor{pcs3}{HTML}{B30086}
\definecolor{pcs4}{HTML}{3C00B3}
\definecolor{pcs5}{HTML}{2A007D}

\definecolorstyle{NewColour} {
  \definecolor{c1}{named}{nbBlack}
  \definecolor{c2}{named}{nbPurple}
  \definecolor{c3}{named}{nbYellow}
}{
  % Background Colors
  \colorlet{backgroundcolor}{black!10}
  \colorlet{framecolor}{black}
  % Title Colors
  \colorlet{titlefgcolor}{black}
  \colorlet{titlebgcolor}{black!10}
  % Block Colors
  \colorlet{blocktitlebgcolor}{c1}
  \colorlet{blocktitlefgcolor}{white}
  \colorlet{blockbodybgcolor}{white}
  \colorlet{blockbodyfgcolor}{black}
  % Innerblock Colors
  \colorlet{innerblocktitlebgcolor}{c2!80}
  \colorlet{innerblocktitlefgcolor}{black}
  \colorlet{innerblockbodybgcolor}{c2!50}
  \colorlet{innerblockbodyfgcolor}{black}
  % Note colors
  \colorlet{notefgcolor}{black}
  \colorlet{notebgcolor}{c3!50}
  \colorlet{notefrcolor}{c3!70}
}

\defineblockstyle{NewBlock}{
  titlewidthscale=1, bodywidthscale=1, titleleft,
  titleoffsetx=0pt, titleoffsety=0pt, bodyoffsetx=0pt, bodyoffsety=0pt,
  bodyverticalshift=0pt, roundedcorners=0, linewidth=0pt, titleinnersep=1cm,
  bodyinnersep=1cm
}{
  \ifBlockHasTitle%
  \draw[draw=none, fill=blocktitlebgcolor]
  (blocktitle.south west) rectangle (blocktitle.north east);
  \fi%
  \draw[draw=none, fill=blockbodybgcolor] %
  (blockbody.north west) [rounded corners=30] -- (blockbody.south west) --
  (blockbody.south east) [rounded corners=0]-- (blockbody.north east) -- cycle;
}

% Choose Layout
\usecolorstyle{NewColour}
\usebackgroundstyle{Default}
\usetitlestyle{Filled}
\useblockstyle{NewBlock}
\useinnerblockstyle[roundedcorners=0.2]{Default}
\usenotestyle[roundedcorners=0]{Default}

\settitle{\centering \color{titlefgcolor} {\Large \@title \, -- \, \@author}}

% Title, Author, Institute
\title{Probability, Measure and Martingales}
\author{Ike Glassbrook}

\begin{document}

\message{ !name(main.tex) !offset(-3) }


% Title block with title, author, logo, etc.
\maketitle[titletoblockverticalspace=0.4cm]
\begin{columns}
  \column{0.5}
  \block{Measurable sets and functions}{
    \begin{definition}[$\sigma$-algebras]
      \ Let $\Omega$ be a set and $\mathcal{A} \subseteq \mathcal{P}(\Omega)$ be a collection of subsets of $\Omega$:
      \begin{enumerate}
              \item $\mathcal{A}$ is an \emph{algebra} if $\varnothing \in \mathcal{A}$ and for $A, B \in \mathcal{A}$, $\mathcal{A}^{c} = \Omega \setminus A \in \mathcal{A}$ and $A \cup B \in \mathcal{A}$.
              \item $\mathcal{A}$ is a \emph{$\sigma$-algebra} if $\varnothing \in \mathcal{A}$, for $A \in \mathcal{A}$, $A^{c} \in \mathcal{A}$, and for $(A_{n})$ a sequence of sets in $\mathcal{A}$, $\bigcup_{n=1}^{\infty} A_{n} \in \mathcal{A}$.
      \end{enumerate}

    \end{definition}
    \hphantom{}

    A collection of sets is an algebra subject to being closed under finite applications of the basic operators. The $\sigma$-algebra concept extends this slightly to infinite ones. Consider where this distinction is relevant? \\

    Note that if we have $\{\mathcal{F}_{i} : i \in I\}$ are $\sigma$-algebras, then
    \begin{align*}
      \mathcal{F} = \bigcap_{i \in I} \mathcal{F}_{i}
    \end{align*}
    is a $\sigma$-algebra. This allows us to consider the notion of a smallest $\sigma$-algebra containing a set (the $\sigma$-algebra `generated' by a set). We write the $\sigma$-algebra generated by a collection of collections of sets $\mathfrak{A}$ as $\sigma(\mathfrak{A})$. \\

    \begin{definition}[Borel $\sigma$-algebra]
    \ Let $(E, \mathcal{T})$ be a topological space. The $\sigma$-algebra generated by the open sets in $E$ is called the \emph{Borel $\sigma$-algebra on $E$} and is denoted $\mathcal{B}(E) = \sigma(\mathcal{T})$.
    \end{definition}
    \hphantom{}

    \begin{definition}
    \ Suppose $(\Omega_{i}, \mathcal{F}_{i})_{i \in I}$ are measurable spaces. With $\Omega = \prod_{i \in I} \Omega_{i}$, $\mathcal{F}$ the $\sigma$-algebra generated by $A = \prod_{i \in I} A_{i}$ where $A_{i} \in \mathcal{F}_{i}$ for all $i \in I$ and for all but finitely many $i \in I$, $A_{i} = \Omega_{i}$: $(\Omega, \mathcal{F})$ is the product space.
    \end{definition}
    \hphantom{}

    This space is measurable, and $\mathcal{F}$ is a $\sigma$-algebra. \\

    \begin{definition}[$\pi$ and $\lambda$-systems]
    \ A collection of sets $\mathcal{A}$ is called a $\pi$-system if it is closed under intersections. \\

      A collection of sets $\mathcal{M}$ is called a $\lambda$-system if $\Omega \in \mathcal{M}$, if $A, B \in \mathcal{M}$, $A \subseteq B$, then $B \setminus A \in \mathcal{M}$, and if $(A_{n}) \subseteq \mathcal{M}$ with $A_{n} \subseteq A_{n+1}$ increasing then $\bigcup_{n \ge 1} A_{n} \in \mathcal{M}$.
    \end{definition}
    \hphantom{}

    A collection of sets is a $\sigma$-algebra if and only if it is both a $\pi$-system and a $\lambda$-system. \\

    \begin{lemma}[$\pi$-$\lambda$ systems lemma]
    \ Let $\mathcal{A}$ be a $\pi$-system and $\mathcal{M}$ a $\lambda$-system. Then if $\mathcal{A} \subseteq \mathcal{M}$ then $\sigma(\mathcal{A}) \subseteq \mathcal{M}$.
    \end{lemma}
    \hphantom{}

    We can use this with a convenient $\pi$-system to show that our $\lambda$-system contains more than is immediately obvious. \\

    Let $\lambda(\mathcal{A})$ be the smallest $\lambda$-system containing $\mathcal{A}$. This is a subset of $\mathcal{M}$ and $\sigma(\mathcal{A})$, so we just need to show that $\lambda(\mathcal{A})$ is a $\sigma$-algebra (for which we just have to show that it is a $\pi$-system). \\

    \begin{definition}[Random variables]
    \ With measurable spaces $(\Omega, \mathcal{F})$, $(E, \mathcal{E})$, a function $f : \Omega \to E$ is said to be an $E$-valued random variable (or a measurable function) if for all $A \in \mathcal{E}$, $f^{-1}(A) \in \mathcal{F}$.
    \end{definition}
    \hphantom{}

    We get immediately that random variables can be composed as one would expect. We can also use random variables to define new $\sigma$-algebras. Note that $(\Omega, \{f^{-1}(A) : A \in \mathcal{E}\})$ is a $\sigma$-algebra. \\

    \begin{definition}
    \ With $\{f_{i} : i \in I\}$ a family of functions $\Omega \to E$, $\sigma(f_{i} : i \in I)$ is the smallest $\sigma$-algebra on $\Omega$ for which all $f_{i}$ are measurable.
    \end{definition}
    \hphantom{}

    This is initially a slightly intimidating definition, but the intuition is just that we need our $\sigma(f_{i} : i \in I) = \sigma(f_{i}^{-1}(A) : A \in \mathcal{E}, i \in I)$. \\

    \begin{theorem}[Monotone Class Theorem]
      \ Let $\mathcal{H}$ be a class of bounded functions from $\Omega \to \mathbb{R}$ such that
      \begin{itemize}
              \item \ $\mathcal{H}$ is a vector space over $\mathbb{R}$,
              \item \ The constant function $1 \in \mathcal{H}$,
              \item \ If $(f_{n}) \subseteq \mathcal{H}$, $f_{n} \to f$ monotonically increasing, then $f \in \mathcal{H}$,
      \end{itemize}
      then if $\mathcal{C} \subseteq \mathcal{H}$, and $\mathcal{C}$ is closed under multiplication, then all $\sigma(\mathcal{C})$-bounded functions are in $\mathcal{H}$.
    \end{theorem}
  }


  \column{0.5}
  \block{Measures on $\mathbb{R}$}{
    \begin{definition}
    \ Let $\mu$ be a probability measure on $\mathcal{B}(\mathbb{R})$. The distribution function of $\mu$ is $F_{\mu}(x) = \mu(-\infty,x]$, where we require that $F_{\mu}$ is non-decreasing, tends to $0$ as $x \to -\infty$, to $1$ as $x \to \infty$, and is right continuous.
    \end{definition}
    \hphantom{}

    \begin{theorem}
    \ Let $F$ be a distribution function. Then there exists a unique Borel probability measure $\mu$ on $\mathbb{R}$ such that $\mu(-\infty,x] = F(x)$. Further, every Borel probability measure on $\mathbb{R}$ defines a distribution function.
    \end{theorem}
    \hphantom{}

    A corollary of this is that there is a unique Borel measure such that for all $a < b \in \mathbb{R}$, $\mu(a,b] = b-a$.
  }
\end{columns}

\end{document}

\message{ !name(main.tex) !offset(-208) }
