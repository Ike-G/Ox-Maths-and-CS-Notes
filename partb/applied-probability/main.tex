\documentclass{tikzposter} %Options for format can be included here
\geometry{paperwidth=850mm, paperheight=2000mm}
\makeatletter
\setlength{\TP@visibletextwidth}{\textwidth-2\TP@innermargin}
\setlength{\TP@visibletextheight}{\textheight-2\TP@innermargin}
\makeatother
\usepackage{amsmath}
\usepackage{dsfont}
\usepackage{parskip}
\usepackage{amsfonts}
\usepackage{amssymb}
\usepackage{verbatim}
\usepackage{enumitem}
\usepackage{nicefrac}
\usepackage{mathtools}
\usepackage{bm}
\DeclarePairedDelimiter{\floor}{\lfloor}{\rfloor}
\DeclarePairedDelimiter{\ceil}{\lceil}{\rceil}
\DeclareMathOperator{\Int}{int}
\DeclareMathOperator{\cl}{cl}
\DeclareMathOperator{\Markov}{Markov}
\DeclareMathOperator{\Span}{span}
\DeclareMathOperator{\dist}{dist}
\DeclareMathOperator{\Exp}{Exp}
\DeclareMathOperator{\Po}{Po}
\DeclareMathOperator{\PP}{PP}

\newcommand\restr[2]{{% we make the whole thing an ordinary symbol
  \left.\kern-\nulldelimiterspace % automatically resize the bar with \right
  #1 % the function
  \vphantom{\big|} % pretend it's a little taller at normal size
  \right|_{#2} % this is the delimiter
  }}
\newcommand\leftopen[2]{\ensuremath{(#1,#2]}}
\newcommand\rightopen[2]{\ensuremath{[#1,#2)}}

\newtheorem{theorem}{Theorem}
\newtheorem{lemma}[theorem]{Lemma}
\newtheorem{corollary}{Corollary}

\newtheorem{definition}{Definition}

\newtheorem{remark}{Remark}
\newtheorem{claim}{Claim}
\newtheorem{case}{Case}

\definecolor{nbYellow}{HTML}{FCF434}
\definecolor{nbPurple}{HTML}{9C59D1}
\definecolor{nbBlack}{HTML}{2C2C2C}
\definecolor{tBlue}{HTML}{5BCEFA}
\definecolor{tPink}{HTML}{F5A9B8}
\definecolor{bp1}{HTML}{D60270}
\definecolor{bp2}{HTML}{9B4F96}
\definecolor{bp3}{HTML}{0038A8}
\definecolor{pcs1}{HTML}{B300B3}
\definecolor{pcs2}{HTML}{54007D}
\definecolor{pcs3}{HTML}{B30086}
\definecolor{pcs4}{HTML}{3C00B3}
\definecolor{pcs5}{HTML}{2A007D}

\definecolorstyle{NewColour} {
  \definecolor{c1}{named}{nbBlack}
  \definecolor{c2}{named}{nbPurple}
  \definecolor{c3}{named}{nbYellow}
}{
  % Background Colors
  \colorlet{backgroundcolor}{black!10}
  \colorlet{framecolor}{black}
  % Title Colors
  \colorlet{titlefgcolor}{black}
  \colorlet{titlebgcolor}{black!10}
  % Block Colors
  \colorlet{blocktitlebgcolor}{c1}
  \colorlet{blocktitlefgcolor}{white}
  \colorlet{blockbodybgcolor}{white}
  \colorlet{blockbodyfgcolor}{black}
  % Innerblock Colors
  \colorlet{innerblocktitlebgcolor}{c2!80}
  \colorlet{innerblocktitlefgcolor}{black}
  \colorlet{innerblockbodybgcolor}{c2!50}
  \colorlet{innerblockbodyfgcolor}{black}
  % Note colors
  \colorlet{notefgcolor}{black}
  \colorlet{notebgcolor}{c3!50}
  \colorlet{notefrcolor}{c3!70}
}

\defineblockstyle{NewBlock}{
  titlewidthscale=1, bodywidthscale=1, titleleft,
  titleoffsetx=0pt, titleoffsety=0pt, bodyoffsetx=0pt, bodyoffsety=0pt,
  bodyverticalshift=0pt, roundedcorners=0, linewidth=0pt, titleinnersep=1cm,
  bodyinnersep=1cm
}{
  \ifBlockHasTitle%
  \draw[draw=none, fill=blocktitlebgcolor]
  (blocktitle.south west) rectangle (blocktitle.north east);
  \fi%
  \draw[draw=none, fill=blockbodybgcolor] %
  (blockbody.north west) [rounded corners=30] -- (blockbody.south west) --
  (blockbody.south east) [rounded corners=0]-- (blockbody.north east) -- cycle;
}

% Choose Layout
\usecolorstyle{NewColour}
\usebackgroundstyle{Default}
\usetitlestyle{Filled}
\useblockstyle{NewBlock}
\useinnerblockstyle[roundedcorners=0.2]{Default}
\usenotestyle[roundedcorners=0]{Default}

\settitle{\centering \color{titlefgcolor} {\Large \@title \, -- \, \@author}}

% Title, Author, Institute
\title{Applied Probability}
\author{Ike Glassbrook}

\begin{document}

% Title block with title, author, logo, etc.
\maketitle[titletoblockverticalspace=0.4cm]
\begin{columns}
  \column{0.5}
  \block{Poisson Processes}{
    \begin{definition}[Markov property]
    \ The sequence of random variables $(Y_{n})$ is a discrete time Markov chain with state space $\mathbb{S}$ if it satisfies the Markov property
    \begin{align*}
      \mathbb{P}(Y_{n} = y_{n} \,|\, Y_{0}=y_{0},\dots,Y_{n-1}=y_{n-1}) &= \mathbb{P}(Y_{n} = y_{n} \,|\, Y_{n-1}=y_{n-1}).
    \end{align*}
    \end{definition}

    As shown in Part A, provided the Markov chain is time-homogeneous, then the distribution of $(Y_{n})$ is entirely specified by an initial distribution $\mu_{s} = \mathbb{P}(Y_{0} = s)$, and a transition matrix $P = (P_{st})_{s,t \in \mathbb{S}}$. \\

    \begin{definition}
    \ Let $(Z_{n})$ be a sequence of i.i.d. $\Exp(\lambda)$ distributed random variables for $\lambda > 0$. With
    \begin{align*}
      T_{n} = \sum_{k=1}^{n} Z_{k},
    \end{align*}
    we define $(X_{t})$ for $t \ge 0$ by
    \begin{align*}
      X_{t} = \sup \{n \ge 1 : T_{n} \le t\}.
    \end{align*}
    $(X_{t})$ is called a Poisson process of rate $\lambda$, written $(X_{t}) \sim \PP(\lambda)$.
    \end{definition}
    \hphantom{}

    \begin{theorem}
    \ Take $(X_{t}) \sim \PP(\lambda)$. For $t \ge 0$, $(X_{r})_{r \le t}$ and $(X_{r})_{r \ge t}$ are conditionally independent on $X_{t}$, and $(X_{s+t})_{s \ge 0} \sim \PP(\lambda)$ started from $X_{t}$. \textbf{rewrite this}.
    \end{theorem}
  }
  \column{0.5}
  \block{Continuous-time Markov chains}{
    In general with continuous stochastic processes, we are most interested in right-continuous minimal processes. If a process explodes, and is set to $\infty$ after the explosion, then it is called minimal.

    \begin{definition}
    \ A $Q$-matrix or generator is a matrix $Q = (q_{ij})$ for $i,j \in \mathbb{S}$ a countable state space, such that
    \begin{enumerate}[label=\roman*.]
            \item for $i \in \mathbb{S}$, $q_{ii} \in \leftopen{\infty}{0}$,
            \item for $i, j \in \mathbb{S}$, $i \neq j$, $q_{ij} \in \rightopen{0}{\infty}$, and
      \item for $i \in \mathbb{S}$, $\displaystyle \sum_{j \in \mathbb{S}} q_{ij} = 0$.
    \end{enumerate}
    We often write $q_{i} := -q_{ii}$.
    \end{definition}
    \hphantom{}

    We construct from this a stochastic matrix $\Pi = (\pi_{ij})$ such that for $q_{i} \neq 0$, $j \in \mathbb{S}$,
    \begin{align*}
      \pi_{ij} := (1-\delta_{ij}) \frac{q_{ij}}{q_{i}}
    \end{align*}
    and for $q_{i} = 0$, $j \in \mathbb{S}$,
    \begin{align*}
      \pi_{ij} = \delta_{ij}.
    \end{align*}
    \hphantom{}


    \begin{definition}[Jump chain]
    \ A minimal right-continuous process $(X_{t})$ is a continuous time Markov chain with initial distribution $\nu$ and $Q$-matrix $Q$, $(X_{t}) \sim \Markov(\nu, Q),$ if
    \begin{enumerate}[label=\roman*.]
            \item $(Y_{n})$ is a discrete Markov chain with initial distribution $\nu$, transition matrix $\Pi$;
            \item conditional on $Y_{0} = i_{0}, Y_{1} = i_{i}, \dots, Y_{n-1} = i_{n-1}$, the holding times $Z_{1},Z_{2},\dots,Z_{n-1}$ are independent exponential random variables distributed with parameters $q_{i_{0}},q_{i_{1}},\dots,q_{i_{n-1}}$; and
      \item with $T_{n} := \sum_{k=1}^{n} Z_{k}$,
            \begin{align*}
              X_{t} = \begin{cases}
                Y_{n} & \text{if } T_{n} \le t < T_{n+1} \\
                \infty & \text{otherwise}.
                \end{cases}
            \end{align*}
    \end{enumerate}
    \end{definition}
    \hphantom{}



  }
\end{columns}

\end{document}
