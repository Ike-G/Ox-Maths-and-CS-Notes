\documentclass[a0paper,blockverticalspace=5pt]{tikzposter} %Options for format can be included here
\usepackage{amsmath}
\usepackage{amsthm}
\usepackage{amsfonts}
\usepackage{amssymb}
\usepackage{verbatim}
\usepackage{nicefrac}
\usepackage{mathtools}
\usepackage{parskip}
\DeclarePairedDelimiter{\floor}{\lfloor}{\rfloor}

\definecolor{nbYellow}{HTML}{FCF434}
\definecolor{nbPurple}{HTML}{9C59D1}
\definecolor{nbBlack}{HTML}{2C2C2C}
\definecolor{tBlue}{HTML}{5BCEFA}
\definecolor{tPink}{HTML}{F5A9B8}
\definecolor{bp1}{HTML}{D60270}
\definecolor{bp2}{HTML}{9B4F96}
\definecolor{bp3}{HTML}{0038A8}
\definecolor{pcs1}{HTML}{B300B3}
\definecolor{pcs2}{HTML}{54007D}
\definecolor{pcs3}{HTML}{B30086}
\definecolor{pcs4}{HTML}{3C00B3}
\definecolor{pcs5}{HTML}{2A007D}

\definecolorstyle{NewColour} {
  \definecolor{c1}{named}{nbBlack}
  \definecolor{c2}{named}{nbPurple}
  \definecolor{c3}{named}{nbYellow}
}{
  % Background Colors
  \colorlet{backgroundcolor}{black!10}
  \colorlet{framecolor}{black}
  % Title Colors
  \colorlet{titlefgcolor}{black}
  \colorlet{titlebgcolor}{black!10}
  % Block Colors
  \colorlet{blocktitlebgcolor}{c1}
  \colorlet{blocktitlefgcolor}{white}
  \colorlet{blockbodybgcolor}{white}
  \colorlet{blockbodyfgcolor}{black}
  % Innerblock Colors
  \colorlet{innerblocktitlebgcolor}{c2!80}
  \colorlet{innerblocktitlefgcolor}{black}
  \colorlet{innerblockbodybgcolor}{c2!50}
  \colorlet{innerblockbodyfgcolor}{black}
  % Note colors
  \colorlet{notefgcolor}{black}
  \colorlet{notebgcolor}{c3!50}
  \colorlet{notefrcolor}{c3!70}
}

\defineblockstyle{NewBlock}{
  titlewidthscale=1, bodywidthscale=1, titleleft,
  titleoffsetx=0pt, titleoffsety=0pt, bodyoffsetx=0pt, bodyoffsety=0pt,
  bodyverticalshift=0pt, roundedcorners=0, linewidth=0pt, titleinnersep=1cm,
  bodyinnersep=1cm
}{
  \ifBlockHasTitle%
  \draw[draw=none, fill=blocktitlebgcolor]
  (blocktitle.south west) rectangle (blocktitle.north east);
  \fi%
  \draw[draw=none, fill=blockbodybgcolor] %
  (blockbody.north west) [rounded corners=30] -- (blockbody.south west) --
  (blockbody.south east) [rounded corners=0]-- (blockbody.north east) -- cycle;
}
%\definenotestyle{NewNote}{
%    targetoffsetx=0pt, targetoffsety=0pt, angle=0, radius=8cm, width=8cm,
%    connection=false, rotate=0, roundedcorners=0, linewidth=0pt, innersep=1cm
%}{
%  \draw[color=notefrcolor, fill=notebgcolor, rounded
%    corners=0] (notecenter.south west) -- (notecenter.north
%    west) -- (notecenter.north east) -- (notecenter.south east) -- cycle;
%}

% Choose Layout
\usecolorstyle{NewColour}
\usebackgroundstyle{Default}
\usetitlestyle{Filled}
\useblockstyle{NewBlock}
\useinnerblockstyle[roundedcorners=0.2]{Default}
% \usenotestyle{NewNote}
\usenotestyle[roundedcorners=0]{Default}

\settitle{\centering \color{titlefgcolor} {\Large \@title \, -- \, \@author}}

% Title, Author, Institute
\title{Probability I}
\author{Ike Glassbrook}

\begin{document}

% Title block with title, author, logo, etc.
\maketitle[titletoblockverticalspace=0.4cm]

\begin{columns}
  \column{0.45}
  \block{Probability Space}{
    Any probability space is a triple $\langle \Omega, \mathcal{F}, \mathbb{P}\rangle$ where
    \begin{itemize}
            \item $\Omega$ is the sample space -- it exists as a set from which $\mathcal{F}$ may be formed.
            \item $\mathcal{F}$ is the set of events -- an event being a subset of $\Omega$, satisfying

            $\mathbf{F}_{1}$ (universal set): $\Omega \in \mathcal{F}$.

            $\mathbf{F}_{2}$ (closed under complementation): $A \in \mathcal{F}$ implies that $\Omega \setminus A \in \mathcal{F}$.

            $\mathbf{F}_{3}$ (closed under countable unions): If $A_{k} \in \mathcal{F}$ for every $k \in \mathbb{N}$, then $\displaystyle \bigcup_{k \in \mathbb{N}} A_{k} \in \mathcal{F}$.

            \coloredbox{Equivalently, $\mathcal{F}$ is a $\sigma$-algebra with respect to $\Omega$. This means $\mathcal{F}$ is measurable, so a non-negative real value may be assigned to each element of $\mathcal{F}$.}
            \item $\mathbb{P}$ is a measure on $\mathcal{F}$ satisfying

            $\mathbf{P}_{1}$ (non-negative): For all $A \in \mathcal{F}$, $\mathbb{P}(A) \ge 0$.

            $\mathbf{P}_{2}$ (1 on whole space): $\mathbb{P}(\Omega) = 1$.

            $\mathbf{P}_{3}$ ($\sigma$-additivity): If $A_{k} \in \mathcal{F}$ for every $k \in \mathbb{N}$, and $A_{i} \cap A_{j} = \varnothing$ for every $i \neq j$, then $\displaystyle\mathbb{P}\left(\bigcup_{k \in \mathbb{N}} A_{k}\right) = \sum_{k \in \mathbb{N}} \mathbb{P}(A_{k})$.
    \end{itemize}
    The first immediate consequence of $\mathbf{P}_{3}$ is the formula for $\mathbb{P}(A \setminus B)$: $\mathbb{P}(A \setminus B) + \mathbb{P}(A \cap B) = \mathbb{P}(A)$. Consequently, $\mathbb{P}(A \cup B) = \mathbb{P}(A) + \mathbb{P}(B) - \mathbb{P}(A \cap B)$. The inclusion-exclusion formula is then derived with $S_{n} = \bigcup_{k=1}^{n} A_{k}$ by induction.

    \innerblock{Random Variables}{
      A discrete random variable is a function from samples to values $X : \Omega \to \mathbb{R}$, such that for every value in $\mathrm{Im } X$ there is an event to which it corresponds (Equivalently, there is a well-defined $f : \mathbb{R} \to \mathcal{F}$ where $f(x) = \{\omega \in \Omega \,|\, X(\omega) = x\}$, $f(\mathbb{R})$ is a partition of $\mathcal{F}$). Additionally, $\mathrm{Im }X$ is countable. \\

      In the case where the random variable is continuous, then the notion of $X : \Omega \to \mathbb{R}$ as a function becomes more general. In this case we redefine $f : \mathbb{R} \to \mathcal{F}$ to $f(x) = \{\omega \in \Omega \,|\, X(\omega) \le x\}$. Note here that $\mathrm{Im }X$ need not be countable. \\

      Given that this occurs in conjunction with $\mathbf{F}_{3}$ and $\mathbf{P}_{3}$, if $\mathcal{F}$ was merely a set of uncountable singletons, then one would not be able to apply $\sigma$-additivity. Instead, we ensure that $\mathcal{F}$ is formed of uncountable sets.
    }
  }
  \column{0.55}
  \block{Difference Equations}{
    A $n$th order difference equation has the form
    \begin{align*}
      \sum_{j=0}^{k}a_{j}u_{n+j} = f(n)
    \end{align*}
    with $a_{0} \neq 0$, $a_{k} \neq 0$, $a_{k}$ independent of $n$ for all $k$. \\

    To solve equations of this form one initially solves the homogeneous equation using the $u_{j} = A\lambda^{j}$ ansatz, then finds a particular solution with respect to $f$. The ansatz gives $k$ solutions, and the auxiliary equation gives an additional solution.

    The set of solutions to any homogeneous $k$th order difference equation is $k$-dimensional.

  }

\end{columns}

\end{document}
