\documentclass{tikzposter} %Options for format can be included here
\usepackage{amsmath}
\usepackage{parskip}
\usepackage{amsfonts}
\usepackage{amssymb}
\usepackage{verbatim}
\usepackage{nicefrac}
\usepackage{mathtools}
\usepackage{enumitem}
\DeclarePairedDelimiter{\floor}{\lfloor}{\rfloor}
\DeclarePairedDelimiter{\ceil}{\lceil}{\rceil}

\newtheorem{theorem}{Theorem}
\newtheorem{lemma}[theorem]{Lemma}
\newtheorem{corollary}{Corollary}

\newtheorem{definition}{Definition}

\newtheorem{remark}{Remark}
\newtheorem{claim}{Claim}
\newtheorem{case}{Case}
\newcommand\restr[2]{{% we make the whole thing an ordinary symbol
  \left.\kern-\nulldelimiterspace % automatically resize the bar with \right
  #1 % the function
  \vphantom{\big|} % pretend it's a little taller at normal size
  \right|_{#2} % this is the delimiter
  }}
\definecolor{nbYellow}{HTML}{FCF434}
\definecolor{nbPurple}{HTML}{9C59D1}
\definecolor{nbBlack}{HTML}{2C2C2C}
\definecolor{tBlue}{HTML}{5BCEFA}
\definecolor{tPink}{HTML}{F5A9B8}
\definecolor{bp1}{HTML}{D60270}
\definecolor{bp2}{HTML}{9B4F96}
\definecolor{bp3}{HTML}{0038A8}
\definecolor{pcs1}{HTML}{B300B3}
\definecolor{pcs2}{HTML}{54007D}
\definecolor{pcs3}{HTML}{B30086}
\definecolor{pcs4}{HTML}{3C00B3}
\definecolor{pcs5}{HTML}{2A007D}

\definecolorstyle{NewColour} {
  \definecolor{c1}{named}{nbBlack}
  \definecolor{c2}{named}{nbPurple}
  \definecolor{c3}{named}{nbYellow}
}{
  % Background Colors
  \colorlet{backgroundcolor}{black!10}
  \colorlet{framecolor}{black}
  % Title Colors
  \colorlet{titlefgcolor}{black}
  \colorlet{titlebgcolor}{black!10}
  % Block Colors
  \colorlet{blocktitlebgcolor}{c1}
  \colorlet{blocktitlefgcolor}{white}
  \colorlet{blockbodybgcolor}{white}
  \colorlet{blockbodyfgcolor}{black}
  % Innerblock Colors
  \colorlet{innerblocktitlebgcolor}{c2!80}
  \colorlet{innerblocktitlefgcolor}{black}
  \colorlet{innerblockbodybgcolor}{c2!50}
  \colorlet{innerblockbodyfgcolor}{black}
  % Note colors
  \colorlet{notefgcolor}{black}
  \colorlet{notebgcolor}{c3!50}
  \colorlet{notefrcolor}{c3!70}
}

\defineblockstyle{NewBlock}{
  titlewidthscale=1, bodywidthscale=1, titleleft,
  titleoffsetx=0pt, titleoffsety=0pt, bodyoffsetx=0pt, bodyoffsety=0pt,
  bodyverticalshift=0pt, roundedcorners=0, linewidth=0pt, titleinnersep=1cm,
  bodyinnersep=1cm
}{
  \ifBlockHasTitle%
  \draw[draw=none, fill=blocktitlebgcolor]
  (blocktitle.south west) rectangle (blocktitle.north east);
  \fi%
  \draw[draw=none, fill=blockbodybgcolor] %
  (blockbody.north west) [rounded corners=30] -- (blockbody.south west) --
  (blockbody.south east) [rounded corners=0]-- (blockbody.north east) -- cycle;
}

% Choose Layout
\usecolorstyle{NewColour}
\usebackgroundstyle{Default}
\usetitlestyle{Filled}
\useblockstyle{NewBlock}
\useinnerblockstyle[roundedcorners=0.2]{Default}
\usenotestyle[roundedcorners=0]{Default}

\settitle{\centering \color{titlefgcolor} {\Large \@title \, -- \, \@author}}

% Title, Author, Institute
\title{Analysis II}
\author{Ike Glassbrook}

\begin{document}

% Title block with title, author, logo, etc.
\maketitle[titletoblockverticalspace=0.4cm]

  \begin{columns}
    \column{0.5}
\block{Function limits}{
  Intuitively we can have a notion of a fuction tending to a limit as its argument approaches a point. To discuss such a notion formally we might state a definition: $\displaystyle\lim_{x \to p} f(x) = l$ if for any arbitrary distance that $f(x)$ might be from $l$, there is a radius around $p$ not including $p$ such that all $f(x)$ are within that distance from $l$. \\

  Without extra work it is difficult to justify this definition, as the $\forall$ quantifier on $x$ could be satisfied simply by the absence of possible $x$ values. Thus we must first define the condition that $p$ is a limit point of the domain of $f$.\\

  \begin{definition}[Limit point]
  \ $p$ is a limit point of $E$ if for all $\delta > 0$ there exists an $x \in E$ such that $0 < |x - p| < \varepsilon$.\\
  \end{definition}

  From this we can define the notions of closure and isolation. More importantly, we can relate the concept to that of sequences: \\

  \begin{claim}[Limit points via sequences]
    A point $p \in \mathbb{R}$ is a limit point of $E \subseteq \mathbb{R}$ if and only if there exists a sequence $(p_{n})$ with $p_{n} \in E \setminus \{p\}$ such that $\displaystyle\lim_{n \to \infty} p_{n} = p$. \\
  \end{claim}

  The proof of this is fairly intuitive -- if $p$ is a limit point then there will exist a sequence of points around it which approach arbitrarily close as $n$ gets large. \\

  Using this we return to the definition of function limits, now more strictly defined as
  \begin{definition}[Function limit]
    Let $E \subseteq \mathbb{R}$ and $f : E \to \mathbb{R}$ be a real-valued function. Let $p$ be a limit point of $E$ and let $l \in \mathbb{R}$. $f$ converges to $l$ as $x$ tends to $p$ if
    \[
      \forall \varepsilon > 0\ \exists \delta > 0\ \forall x \in E\ (0 < |x-p| < \delta \Rightarrow |f(x)-l|< \varepsilon)
    \]
  \end{definition}

  and we can continue the relation to sequences by showing that $\displaystyle\lim_{x\to p} f(x)= l$ is equivalent to the claim that for all $p_{n}$, $\displaystyle \lim_{n \to \infty} f(p_{n}) = l$ if $p_{n} \in E \setminus \{p\}$, and $p_{n} \to p$.

  \coloredbox{
    Whenever the notion of tending to infinity is considered, we may replace within our definitions $\forall \varepsilon > 0 \hdots \Rightarrow |f(x)-l| < \varepsilon$ with $\forall M \hdots \Rightarrow f(x) > M$. \\

    This applies in a similar way when the limit is tending to infinity: we replace $\exists \delta > 0 \ \hdots \ 0 < |x - p| < \delta \Rightarrow \hdots$ with $\exists N \hdots \ x > N \Rightarrow \hdots$ (or vice versa when tending to negative infinity).
  }
  \phantom{}
  \innerblock{Properties}{
    \begin{theorem}[AOL for functions]
      Let $E \subseteq \mathbb{R}$ and let $p$ be a limit point of $E$. Let $f, g : E \to \mathbb{R}$ and suppose that $f(x) \to a$ and $g(x) \to b$ as $x \to p$. Then
      \begin{itemize}
              \item $|f(x)| \to |a|$,
              \item $f(x)+g(x) \to a + b$,
              \item $f(x)-g(x) \to a - b$,
              \item $f(x)g(x) \to ab$ and
              \item if $b \neq 0$ then $f(x) / g(x) \to a/b$
      \end{itemize}
    \end{theorem}
    These statements all follow from AOL. Extensions of this AOL then come from considering infinite limits and arguments, although these are tedious to prove. \\

    Additionally we get the preservation of weak inequalities, sandwiching, and the limits of composition of functions. That is, if $f(x) \to q$ for $x \to p$, but $f(x) \neq q$ (this is required to ensure $q$ is a limit point) for all $x \in E \setminus \{p\}$, and $g(y) \to l$ as $y \to q$, thus $g(f(x)) \to l$ as $x \to p$.\\

    The requirement of ``$\forall x \in E$'' may be loosened by the observation that for many $p$ only a subset of $E$ is required.
  }
  \phantom{}
  \innerblock{Continuity}{
    \begin{definition}[Continuity]
      Let $f: E \to \mathbb{R}$, where $E \subseteq \mathbb{R}$ and $p \in E$. $f$ is continuous at $p$ if
      \[
        \forall \varepsilon > 0 \ \exists \delta > 0 \ \forall x \in E \ (|x - p| < \delta \Rightarrow |f(x) - f(p)| < \varepsilon)
      \]
    \end{definition}
    or equivalently: $f$ is continuous if $p$ is not a limit point of $E$, and otherwise iff $\displaystyle\lim_{x \to p}f(x) = f(p)$. \\
    \begin{lemma}[Continuity via restrictions]
      For $f : E \to \mathbb{R}$, if there exists $I$ of which $x$ is an element of the interior such that $\restr{f}{I}$ is continuous, then $f$ is continuous at $x$.
    \end{lemma}
  }
}
\block{Convergence}{
  \begin{theorem}[Uniformity preserves continuity]
    Let $(f_{n})$ be a sequence of continuous functions on $E$ which converges uniformly to $f$ on $E$. Then $f$ is continuous on $E$.
  \end{theorem}
  Proof: For any finite $n$, $\delta$-close $x$ and $y$ have mappings $\varepsilon$-close by $f_{n}$. We also have that $|f(p)-f_{n}(p)| < \varepsilon$ for any $p$, using the same lower bound on $n$. Thus we find that $f(x)$ and $f(y)$ must be $3\varepsilon$-close.\\

  \begin{lemma} \
    $\displaystyle f_{n} \overset{u}{\to} f \Leftrightarrow \lim_{n \to \infty} \sup_{x \in E} |f_{n}(x)-f(x)| = 0$. \\
  \end{lemma} \\
  \begin{theorem}[Weierstrass' M-test] \
    Suppose there exist real constants $M_{k}$ such that for all $k \in \mathbb{N}$, $x \in E$, $|u_{k}(x)| \le M_{k}$ and $\sum M_{k}$ converges. Then $\sum u_{k}(x)$ converges uniformly on $E$.
  \end{theorem}

}
\column{0.5}
\block{Boundedness and IVT}{
  \begin{theorem}[Boundedness Theorem]
    Suppose $a < b$ and $f : [a, b] \to \mathbb{R}$ is a continuous function. Then
    \begin{itemize}
      \item $f$ is bounded.
      \item $f$ attains its $\sup$ and its $\inf$. That is, there exist points $\xi_{1}$ and $\xi_{2}$ in $[a,b]$ such that $\displaystyle f(\xi_{1}) = \sup_{x \in [a,b]} f(x)$ and $\displaystyle f(\xi_{2}) = \inf_{x \in [a,b]} f(x)$.
    \end{itemize}
  \end{theorem}
  The proof of the former statement comes by contradiction - if $f$ is unbounded then we have both for any $n$ there exists $x_{n} \in [a,b]$ such that $|f(x_{n})| > n$, and by Bolzano-Weierstrass $(x_{n})$ has a convergent subsequence in $[a,b]$. As $f$ is continuous thus $f(x_{s_{n}})$ is also convergent and thus bounded. This creates a contradiction however, as $|f(x_{s_{n}})| > s_{n} \ge n$. \\

  The second statement is proved by setting $\displaystyle M = \sup_{x \in [a,b]} f(x)$, and using the approximation property of the supremum combined with the closed property of the interval, Bolzano Weierstrass, and sandwiching. \\

  \begin{theorem}[IVT]
    Assume $a < b$, $f : [a,b] \to \mathbb{R}$ is continuous, and $c \in (f(a),f(b))$. Then there is at least one point $\xi \in [a,b]$ such that $f(\xi) = c$.
  \end{theorem}
  Proof: divide and conquer, or using inf and sup.
  \begin{theorem}
    Let $f : [a,b] \to \mathbb{R}$ be continuous. Then $f([a,b]) = [c,d]$ for some $c,d \in \mathbb{R}$
  \end{theorem}
}

\block{MVT}{
    \begin{theorem}[MVT] \
      Let $a < b$ and $f : [a,b] \to \mathbb{R}$. Assume that $f$ is continuous on $[a,b]$ and differentiable on $(a,b)$. Thus there exists some $\xi \in (a,b)$ such that
      \[
        f'(\xi) = \frac{f(b)-f(a)}{b-a}
      \]
    \end{theorem}
    Proof: Define
    \begin{align*}
      F(x) &= f(x)(b-a)-x(f(b)-f(a))+af(b)-bf(a), \\
      \text{thus} \ F'(x) &= f'(x)(b-a)-(f(b)-f(a)) \\
      F(a) &= f(a)(b-a+a-b)-f(b)(a-a) = 0 \\
      F(b) &= f(b)(b-a-b+a)+f(a)(b-b) = 0
    \end{align*}
    Then we apply Rolle's theorem to achieve the result.
    \begin{theorem}[Cauchy MVT] \
      Let $a < b$ and $f,g : [a,b] \to \mathbb{R}$. Assume $f,g$ are continuous on $[a,b]$, and differentiable on $(a,b)$. Then there exist $\xi \in (a,b)$ such that
      \[
        f'(\xi)(g(b)-g(a)) = g'(\xi)(f(b)-f(a)).
      \]
      If $g'(x) \neq 0$ for all $x \in (a,b)$, then $g(b) \neq g(a)$ and we have
      \[
        \frac{f'(\xi)}{g'(\xi)} = \frac{f(b)-f(a)}{g(b)-g(a)}
      \]
    \end{theorem}
    Proof: Define $F(x) = f(x)(g(b)-g(a))-g(x)(f(b)-f(a))-f(a)g(b)+f(b)g(a)$. By Rolle's theorem there exists $\xi \in (a,b)$ such that $f'(\xi)(g(b)-g(a)) = g'(\xi)(f(b)-f(a))$.
}
\block{L'Hopital's Rule}{
  \begin{theorem}[Simple form]
    Let $f, g : E \to \mathbb{R}$ with $p \in E$ a limit point of $E$. Assume that $f(p) = g(p) = 0$, $f$ and $g$ are differentiable at $p$, and $g'(p) \neq 0$. Then
    \[
      \lim_{x \to p} \frac{f(x)}{g(x)} = \frac{f'(p)}{g'(p)}
    \]
  \end{theorem}
  This follows from algebraic manipulation of the LHS.\\
  \begin{theorem}[\,$\frac{0}{0}$ form] \hphantom{} \\
      Let $f, g : (a,a+\delta) \to \mathbb{R}$ for $a \in \mathbb{R}$, $\delta > 0$. If
      \begin{enumerate}[label=\alph*)]
        \item $f$ and $g$ are differentiable in $(a, a+\delta)$;
        \item $\displaystyle\lim_{x\to a^{+}} f(x) = \lim_{x\to a^{+}} g(x) = 0$;
        \item $g'(x) \neq 0$ for all $x \in (a,a+\delta)$;
        \item $\displaystyle\lim_{x \to a^{+}} \frac{f'(x)}{g'(x)} \in \mathbb{R} \cup \{\pm \infty\}$.
      \end{enumerate}
    Then $g(x) \neq 0$ for any $x \in (a, a+\delta)$ (extend $g$ to $a$ then apply Taylor series), and
    \[
      \lim_{x \to a^{+}} \frac{f(x)}{g(x)} = \lim_{x \to a^{+}} \frac{f'(x)}{g'(x)}
    \]
  \end{theorem}
  Proof: Straightforward via C-MVT. \\
  \begin{theorem}[\,$\frac{\infty}{\infty}$ form] \hphantom{} \\
    Let $f, g : (a, a+\delta) \to \mathbb{R}$ for some $a \in \mathbb{R}$, $\delta > 0$. Assume that
    \begin{enumerate}[label=\alph*)]
      \item $f$ and $g$ are differentiable in $(a,a+\delta)$;
      \item $\displaystyle\lim_{x\to a^{+}} |f(x)| = \lim_{x \to a^{+}} |g(x)| = \infty$;
      \item $g'(x) \neq 0$ on $(a,a+\delta)$;
      \item $\displaystyle \lim_{x \to a^{+}} \frac{f'(x)}{g'(x)}$ exists.
    \end{enumerate}
    Then
    \[
      \lim_{x \to a^{+}} \frac{f(x)}{g(x)} = \lim_{x \to a^{+}} \frac{f'(x)}{g'(x)}.
    \]
  \end{theorem}
  Unfortunately this does not immediately follow from applying the above with $\frac{1}{f}$ and $\frac{1}{g}$, as we don't immediately have that the existence of $\displaystyle \lim_{x \to a^{+}} \frac{f'(x)}{g'(x)}$ implies the existence of $\displaystyle \lim_{x \to a^{+}} \frac{f'(x)}{g'(x)} \left(\frac{g(x)}{f(x)}\right)^{2}$. Instead use Rolle's and C-MVT to get an $\varepsilon$-$\delta$ proof for the LHS.
}
  \end{columns}
\end{document}
